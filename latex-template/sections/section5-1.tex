%% SUBSECTION 5.1 %%
\subsection*{Limitations of Existing Research.}

Research on the use of guided imagery and progressive muscle relaxation to achieve stress reduction and relaxation is compelling but has significant limitations. Psychotherapy groups that implement guided imagery and progressive muscle relaxation are typically homogeneous, time limited, and brief \citep{yalom2005theory}. Relaxation training in group psychotherapy typically includes only one or two group meetings focused on these techniques \citep{yalom2005theory}; thereafter, participants are usually expected to practice the techniques by themselves \citep[see][]{menzies2012imagery}. Future research should address how these relaxation techniques can assist people in diverse groups and how the impact of relaxation techniques may be amplified if treatments are delivered in the group setting over time.

Future research should also examine differences in inpatient versus outpatient psychotherapy groups as well as structured versus unstructured groups. The majority of research on the use of guided imagery and progressive muscle relaxation with psychotherapy groups has used unstructured inpatient groups (e.g., groups in a hospital setting). However, inpatient and outpatient groups are distinct, as are structured versus unstructured groups, and each format offers potential advantages and limitations \citep{yalom2005theory}. For example, an advantage of an unstructured group is that the group leader can reflect the group process and focus on the “here and now,” which may improve the efficacy of the relaxation techniques \citep{yalom2005theory}. However, research also has supported the efficacy of structured psychotherapy groups for patients with a variety of medical, psychiatric, and psychological disorders (Hashim \& Zainol, \citeyear{hashim2015distress}; see also \cite{baider1994progressive}; Cohen \& Fried, \citeyear{cohen2007comparing}). Empirical research assessing these interventions is limited, and further research is recommended.
