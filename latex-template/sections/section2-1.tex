%% SUBSECTION 2.1 %%
\subsection*{Features of Guided Imagery.}

Guided imagery involves a person visualizing a mental image and engaging each sense (e.g., sight, smell, touch) in the process. Guided imagery was first examined in a psychological context in the 1960s, when the behavior theorist Joseph Wolpe helped pioneer the use of relaxation techniques such as aversive imagery, exposure, and imaginal flooding in behavior therapy \citep{achterberg1985imagery, utay2006review}. Patients learn to relax their bodies in the presence of stimuli that previously distressed them, to the point where further exposure to the stimuli no longer provokes a negative response \citep{achterberg1985imagery}.

Contemporary research supports the efficacy of guided imagery interventions for treating medical, psychiatric, and psychological disorders \citep{utay2006review}. Guided imagery is typically used to pursue treatment goals such as improved relaxation, sports achievement, and pain reduction. Guided imagery techniques are often paired with breathing techniques and other forms of relaxation, such as mindfulness \citep[see][]{freebird2012progressive}. The evidence is sufficient to call guided imagery an effective, evidence-based treatment for a variety of stress-related psychological concerns \citep{utay2006review}.
