%% SUBSECTION 2.2 %%
\subsection*{Guided Imagery in Group Psychotherapy.}

Guided imagery exercises improve treatment outcomes and prognosis in group psychotherapy contexts \citep{skovholt1987imagery}. \citet{lange1982guided} underscored two such benefits by showing (a) the role of the group psychotherapy leader in facilitating reflection on the guided imagery experience, including difficulties and stuck points, and (b) the benefits achieved by social comparison of guided imagery experiences between group members. Teaching techniques and reflecting on the group process are unique components of guided imagery received in a group context \citep{yalom2005theory}.

Empirical research focused on guided imagery interventions supports the efficacy of the technique with a variety of populations within hospital settings, with positive outcomes for individuals diagnosed with depression, anxiety, and eating disorders \citep{utay2006review}. Guided imagery and relaxation techniques have even been found to “reduce distress and allow the immune system to function more effectively” \citep[p.~850]{trakhtenberg2008imagery}. For example, \citet{holden-lund1988imagery} examined effects of a guided imagery intervention on surgical stress and wound healing in a group of 24 patients. Patients listened to guided imagery recordings and reported reduced state anxiety, lower cortisol levels following surgery, and less irritation in wound healing compared with a control group. \citeauthor{holden-lund1988imagery} concluded that the guided imagery recordings contributed to improved surgical recovery. It would be interesting to see how the results might differ if guided imagery was practiced continually in a group context.

\begin{figure*}[t]
    \centering
    \includegraphics[width=.45\textwidth]{example-image}
    \caption{
        \dummy{Image caption}.\\
        \indent\emph{Note}. From {\color{RedOrange} or} Adapted from \qq{\dummy{Article Title},} by \dummy{Initials}.~\dummy{Last Name}, \dummy{Year}, \emph{\dummy{Journal Name}, \dummy{Volume}}(\dummy{Issue}), p.~\dummy{Page Number} (\dummy{URL} {\color{RedOrange} or} \dummy{DOI}). \dummy{Copyright Statement}.
        }
    \label{fig: large-image}
\end{figure*}

Guided imagery has also been shown to reduce stress, length of hospital stay, and symptoms related to medical and psychological conditions \citep{scherwitz2005therapy}. For example, \citet{ball2003imagery} conducted guided imagery in a group psychotherapy format with 11 children (ages 5–18) experiencing recurrent abdominal pain. Children in the treatment group (n = 5) participated in four weekly group psychotherapy sessions where guided imagery techniques were implemented. Data collected via pain diaries and parent and child psychological surveys showed that patients reported a 67\% decrease in pain. Despite a small sample size, which contributed to low statistical power, the researchers concluded that guided imagery in a group psychotherapy format was effective in reducing pediatric recurrent abdominal pain.

However, in the majority of guided imagery studies, researchers have not evaluated the technique in the context of traditional group psychotherapy. Rather, in these studies participants usually met once in a group to learn guided imagery and then practiced guided imagery individually on their own \citep[see][for more]{menzies2012imagery}. Thus, it is unknown whether guided imagery would have different effects if implemented on an ongoing basis in group psychotherapy.
